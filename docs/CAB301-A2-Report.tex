\documentclass[12pt,a4paper]{article}
\usepackage[utf8]{inputenc}
\usepackage{graphicx}
\usepackage[none]{hyphenat}
\usepackage{hyperref}
\usepackage{xcolor}
\usepackage{pgfplots}
\usepackage{pgfplotstable}
\pgfplotsset{compat=1.17}

\definecolor{darkgreen}{RGB}{0, 150, 0}
\pgfplotstableread[col sep=comma]{../analysis_results.csv}\datatable

\begin{document}
	\begin{titlepage}
		
		\begin{center}
			\includegraphics[width=0.5\textwidth]{QUT.jpg}\\
			[0.03\textheight]  
			\Large\textbf{Bachelor of IT (Computer Science)}\\
			\Large\textbf{Assignment 2}\\
			\large\textbf{CAB301 - Algorithms and Complexity}\\
			[0.02\textheight]
			\large\textsl{Dane Madsen}\\
			\large\textsl{n10983864@qut.edu.au}
		\end{center}
		
	\end{titlepage}
	\tableofcontents
	\newpage
	
	\section{NoDVDs Method Design and Analysis}
		\subsection{Algorithm Design}
			This method calculates the total number of DVDs in a MovieCollection 
			through a in-order node tree traversal using a stack of BTreeNodes. Firstly, 
			it initializes the DVD count to be 0, creates an empty stack of BTreeNodes 
			and starts traversing from the root. Next, an outer while loop starts and 
			continues until the current BTreeNode is null and the stack count is equal 
			to 0. Inside the outer while loop, the inner while loop traverses the left 
			subtree, pushing left child BTreeNodes onto the stack. After traversing the 
			left subtree, the method pops the first movie from the top of the stack, adds 
			its total copies to the running total, and moves to the right child of the popped 
			movie. This process visits each node, cumulatively adding the total copies to 
			the count. Once the traversal finishes, it returns the total DVDs.\\
		
			\textbf{ALGORITHM} \textit{NoDVDs()}\\
			\null\hspace{1cm}// Calculates the total number of DVDs\\
			\null\hspace{1cm}// Returns the total number of DVDs in the MovieCollection\\
			\null\hspace{1cm}\textit{totalDVDs} $\leftarrow 0$\\
			\null\hspace{1cm}\textit{stack} $\leftarrow$ an empty BTreeNodes stack\\
			\null\hspace{1cm}\textit{curr} $\leftarrow$ root node of the MovieCollection\\
			\null\hspace{1cm}\textbf{while} \textit{curr} $\neq$ null \textbf{or} 
			\textit{stack} \textbf{is not} empty\\
			\null\hspace{2cm}\textbf{while} \textit{curr} $\neq$ null\\
			\null\hspace{3cm}\textit{stack.push(curr)}\\
			\null\hspace{3cm}\textit{curr} $\leftarrow$ \textit{curr.LChild}\\
			\null\hspace{2cm}\textit{curr} $\leftarrow$ \textit{stack.pop()}\\
			\null\hspace{2cm}\textit{totalDVDs} $\leftarrow$ \textit{totalDVDs} + 
			\textit{curr.Movie.TotalCopies}\\
			\null\hspace{2cm}\textit{curr} $\leftarrow$ \textit{curr.RChild}\\
			\null\hspace{1cm}\textbf{return} \textit{totalDVDs}

		\newpage

		\subsection{Algorithm Analysis}
			To perform an emperical analysis of the NoDVDs method, I created 20 
			MovieCollections of exponentially increasing sizes (from 1 to 524388 
			movies which is roughly close to the amount of movies on IMDb). Then, 
			for each of these MovieCollections, I ran the NoDVDs method 10 times 
			and averaged the time taken to run the method. The results of this 
			analysis are shown in Figure 1.\\

			\begin{figure}[h]
				\centering
				\begin{tikzpicture}
					\begin{axis}[
						title={Average Execution Time of NoDVDs()},
						xlabel={Collection Size},
						ylabel={Execution Time (ms)},
						xmode=log,
						ymode=log,
						log ticks with fixed point,
						grid=major,
						xmin=1, xmax=524288,
						ymin=0.0001, ymax=20,
						width=\textwidth,
						height=8cm
					]
					\addplot[mark=*,blue] table[x=MovieCollections,y=AveragedExecutionTime] from \datatable;
					\end{axis}
				\end{tikzpicture}
				\caption{Average Execution Time of NoDVDs()}
			\end{figure}
			
			As Figure 1 shows, though some of the first few MovieCollections have 
			timings that are outliers, after the third MovieCollection (which has 8 values), the average 
			execution times of the NoDVDs method increase linearly. As such the closest efficiency 
			classification for this method would be $O(n)$, where $n$ is the number of movies in the
			given MovieCollection. This hypothesis is can be supported by ploting the $T/n$ vs. $n$ as 
			shown in Figure 2.\\

			\newpage

			\begin{figure}[h]
				\centering
				\begin{tikzpicture}
				  \begin{axis}[
					title={Average Execution Time of NoDVDs() / n},
					xlabel={MovieCollection ID (n)},
					ylabel={Execution Time (T) / n [ms]},
					xmode=log,
					ymode=log,
					log ticks with fixed point,
					grid=major,
					width=\textwidth,
					height=8cm
				  ]
				  \addplot[mark=*,blue] table[x=MovieCollections,y=AveragedTN] from \datatable;
				  \end{axis}
				\end{tikzpicture}
				\caption{Plot of MovieCollection ID (n) vs. Execution Time (T) / n}
				\label{fig:execution_time_over_n_plot}
			\end{figure}
			
			Similar to Figure 1, Figure 2 shows that the first few MovieCollections have 
			$T/n$ values that are outliers, but again after the third MovieCollection, the $T/n$ 
			values become quite stable which supports the hypothesis that the algorithm used 
			for the NoDVDs method has an efficiency classification of $O(n)$.\\
		
		\newpage
	
	\section{Testing}
		\subsection{CompareTo Method Testing}
			To test the CompareTo method I created three XUnit tests. The first test 
			ensures that the method returns -1 when the movie CompareTo is being called 
			on is alphabetically less than the movie being provided to CompareTo as a 
			parameter. It achives this by creating two movies, one with the letter A as 
			a title and the other with the letter B as a title. Then the unit test calls 
			CompareTo on the first movie (with the letter A) and passes the second movie 
			(with the letter B) as a parameter and asserts that the returned value is equal 
			to -1.\\
			\\
			The second test ensures that the CompareTo method returns 0 when the movie 
			CompareTo is being called on is equal to the movie being provided to CompareTo 
			as a parameter. It achives this by creating two movies, both with the letter A 
			as a title. Then the unit test calls CompareTo on the first movie and passes the 
			second movie as a parameter, then it asserts that the returned value is equal to 
			0.\\
			\\
			The third and final test ensures that the CompareTo method returns 1 when the 
			movie CompareTo is being called on is alphabetically greater than the movie 
			being provided to CompareTo as a parameter. It achives this by creating two 
			movies, one with the letter B as a title and the other with the letter A as a 
			title (the same as the first test only in reverse). Then the unit test calls 
			CompareTo on the first movie (with the letter B) and passes the second movie 
			(with the letter A) as a parameter and asserts that the returned value is equal 
			to 1.\\
			
		\newpage

		\subsection{ToString Method Testing}
			To test the ToString method I created three XUnit tests. The first test 
			ensures that the method returns the correct string when the movie has 
			only one copy. It achives this by creating a movie with the title "A", 
			the genre Action, the classification G, the duration 1 and the totalcopies 
			also 1. Then the unit test calls ToString on the movie and asserts that 
			the returned value is equal to the expected string. This test can also be 
			considered the minimum boundary test.\\
			\\
			The second test ensures that the ToString method returns the correct 
			string when the movie has more than one copy. It achives this by creating 
			a movie with the title "Sophies Choice", the genre Drama, the classification 
			M, the duration 151 and the totalcopies 12137123. Then the unit test then 
			calls ToString on the movie and asserts that the returned value is equal 
			to the expected string.\\
			\\
			The third and final test ensures that the ToString method returns the 
			correct string when the movie has a very large number of copies. It 
			achives this by creating a movie with the title "ZZZZZZZZZ", the genre 
			Western, the classification M15Plus, and the duration and totalcopies 
			both 2147483647. Then the unit test calls ToString on the movie and 
			asserts that the returned value is equal to the expected string. This 
			test can also be considered the maximum boundary test.\\

		\newpage

		\subsection{IsEmpty Method Testing}
			

		\subsection{NoDVDs Method Testing}
			To test the NoDVDs method I created three XUnit tests. The first test 
			ensures that the method returns 0 when the MovieCollection is empty. 
			It achives this by creating an empty MovieCollection and asserting that 
			when the NoDVDs method is called on the MovieCollection, it returns 0. 
			This can also be considered the minimum boundary test.\\ 
			\\
			The second test ensures that NoDVDs returns the correct number of DVDs under 
			normal circumstances. To achive this the unit test creates a MovieCollection 
			and then populates it with two movies each with 1000 totalDVDs. Then, the 
			unit test asserts that the the returned value of NoDVDs is equal to 2000.\\
			\\
			The third and final test ensures that the NoDVDs method returns the correct 
			number of DVDs when the amount of DVDs is very large. To achieve this, the 
			unit test creates a MovieCollection and then populates it with two movies 
			each with half the maximum integer value (2147483647) minus one (1073741823, 
			because the maxint is odd) as their totalDVDs. Then, the unit test asserts 
			that the returned value of NoDVDs is equal to the maximum integer value 
			minus one.\\ 


			

\end{document}