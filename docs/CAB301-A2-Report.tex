\documentclass[12pt,a4paper]{article}
\usepackage[utf8]{inputenc}
\usepackage{graphicx}
\usepackage[none]{hyphenat}
\usepackage{hyperref}
\usepackage{xcolor}
\usepackage{pgfplots}
\usepackage{pgfplotstable}
\pgfplotsset{compat=1.17}

\definecolor{darkgreen}{RGB}{0, 150, 0}

\begin{document}
	\begin{titlepage}
		
		\begin{center}
			\includegraphics[width=0.5\textwidth]{QUT.jpg}\\
			[0.03\textheight]  
			\Large\textbf{Bachelor of IT (Computer Science)}\\
			\Large\textbf{Assignment 2}\\
			\large\textbf{CAB301 - Algorithms and Complexity}\\
			[0.02\textheight]
			\large\textsl{Dane Madsen}\\
			\large\textsl{n10983864@qut.edu.au}
		\end{center}
		
	\end{titlepage}
	\tableofcontents
	\newpage
	
	\section{NoDVDs Method}
		\subsection{Algorithm Design}
			This method calculates the total number of DVDs in a MovieCollection 
			through a in-order node tree traversal using a stack of BTreeNodes. Firstly, 
			it initializes the DVD count to be 0, creates an empty stack of BTreeNodes 
			and starts traversing from the root. Next, an outer while loop starts and 
			continues until the current BTreeNode is null and the stack count is equal 
			to 0. Inside the outer while loop, the inner while loop traverses the left 
			subtree, pushing left child BTreeNodes onto the stack. After traversing the 
			left subtree, the method pops the first movie from the top of the stack, adds 
			its total copies to the running total, and moves to the right child of the popped 
			movie. This process visits each node, cumulatively adding the total copies to 
			the count. Once the traversal finishes, it returns the total DVDs.\\
		
			\textbf{ALGORITHM} \textit{NoDVDs()}\\
			\null\hspace{1cm}// Calculates the total number of DVDs\\
			\null\hspace{1cm}// Returns the total number of DVDs in the MovieCollection\\
			\null\hspace{1cm}\textit{totalDVDs} $\leftarrow 0$\\
			\null\hspace{1cm}\textit{stack} $\leftarrow$ an empty BTreeNodes stack\\
			\null\hspace{1cm}\textit{curr} $\leftarrow$ root node of the MovieCollection\\
			\null\hspace{1cm}\textbf{while} \textit{curr} $\neq$ null \textbf{or} 
			\textit{stack} \textbf{is not} empty\\
			\null\hspace{2cm}\textbf{while} \textit{curr} $\neq$ null\\
			\null\hspace{3cm}\textit{stack.push(curr)}\\
			\null\hspace{3cm}\textit{curr} $\leftarrow$ \textit{curr.LChild}\\
			\null\hspace{2cm}\textit{curr} $\leftarrow$ \textit{stack.pop()}\\
			\null\hspace{2cm}\textit{totalDVDs} $\leftarrow$ \textit{totalDVDs} + 
			\textit{curr.Movie.TotalCopies}\\
			\null\hspace{2cm}\textit{curr} $\leftarrow$ \textit{curr.RChild}\\
			\null\hspace{1cm}\textbf{return} \textit{totalDVDs}

		\newpage

		\subsection{Algorithm Analysis}
			To perform an emperical analysis of the NoDVDs method, I created 20 
			MovieCollections of exponentially increasing sizes (from 1 to 524388 
			movies). Then, for each of these MovieCollections, I ran the NoDVDs 
			method 10 times and averaged the time taken to run the method. The 
			results of this analysis are shown in Figure 1.\\

			\begin{figure}[h]
				\centering
				\begin{tikzpicture}
					\begin{axis}[
						title={Average Execution Time of NoDVDs()},
						xlabel={Collection Size},
						ylabel={Execution Time (ms)},
						xmode=log,
						ymode=log,
						log ticks with fixed point,
						grid=major,
						xmin=1, xmax=524288,
						ymin=0.0001, ymax=20,
						width=\textwidth,
						height=8cm
					]
					% Read the data from the CSV file
					\pgfplotstableread[col sep=comma]{../analysis_results.csv}\datatable
					% Plot the data from the datatable
					\addplot[mark=*,blue] table[x=MovieCollections,y=AveragedExecutionTime] from \datatable;
					\end{axis}
				\end{tikzpicture}
				\caption{Average Execution Time of NoDVDs()}
			\end{figure}
			
			As this figure shows, though some of the first few MovieCollections have 
			timings that are outliers, after the third MovieCollection (with 8 values), the average 
			execution times of the NoDVDs method increase linearly.
			

			

\end{document}